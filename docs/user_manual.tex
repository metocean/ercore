\documentclass[a4paper]{article}
\renewcommand{\familydefault}{\sfdefault}
\usepackage[english]{babel}
\usepackage[utf8x]{inputenc}
\usepackage{amsmath}
\usepackage{graphicx}
\usepackage{placeins} % FloatBarrier
\usepackage{hyperref}
\usepackage{booktabs}

\usepackage{color}
\usepackage{soul}

\usepackage{fancyvrb}
\usepackage{parskip} % remove indent start of paragrapha and add space


\title{ERCore User Manual}
\author{MetOcean Solutions Ltd.}

\begin{document}
\maketitle

\section{Introduction}

This is a user manual. For techincal description see the other document.

ERCore is a lagrangian model that for every model time step computes positions of a number of particles. Each particle can have the following status:

\begin{itemize}
\item 0: Not released
\item 1: Released and active
\item -1: Stuck to shoreline or bottom
\item -2: Dead
\end{itemize}
 
whereas status 0 and -2 will \hl{never} appear in the output files.

For each model time step, new particle positions are computed from the "active" pool (status 1) and stored in an array with ($\texttt{nbuff}\times 3$)  where colums are x, y and z coordinates. This buffef array should be big enough to accomodate all particles in the computational pool, but small enough to maintain memory and performance. With that in mind, the model reuses array position of "dead" particles (status -2).  \texttt{nbuff} is defined for each release (see table \ref{tb:material}). 

\subsection{Coordinates}
\label{ssec:coords}

ERCore can use any system of coordinates, provided they are consistent in all input and configuraton data. See zinvert = True..


\subsection{Date and time format}
\label{ssec:datetime}

Internally, the model uses time in NCEP/CF convention decimal time (\hl{matlab time?}) which is the "number of days since 1-1-1"  and can be computed with:

\begin{Verbatim}[fontsize=\small]
netCDF4.date2num(t0, units='days since 0001-01-01 00:00:00', calendar='standard') 
\end{Verbatim}

or 

\begin{Verbatim}[fontsize=\small]
_DT0_=datetime.datetime(2000,1,1)
_NCEPT0_=730120.99999
ncep2dt=lambda t:_DT0_+datetime.timedelta(t-_NCEPT0_)
dt2ncep=lambda t: (1.+t.toordinal()+t.hour/24.+t.minute/1440.+t.second/86400.)
\end{Verbatim}


Input dates can be either:

\begin{itemize} 
\item CF decimal time
\item datetime python objects, or 
\item strings like "\%Y\%m\%d\_\%Hz" or "\%Y-\%m-\%d \%H:\%M:\%S".
\end{itemize} 



\section{Materials}
\label{sec:materials}


ERCore allows for releases of different particle types, called "materials". The base class, from which all materials inherit, defines the basic options for a release, as listed in table~\ref{tb:material}. 

\begin{table}[!htp]
\centering
\caption{Common options for all materials. *Note: particle vertical level Z is positive upwards with sea surface = 0, i.e. -10 is 10 m below sea surface.}
\label{tb:material}
\begin{tabular}{@{}llll@{}}
\toprule
Keyword & Type & Default & Description                         \\ 
\midrule
id          & str  &        & Unique id for release                \\
outfile     & str & None & Filename of output file \\
P0          & [float,float,float] & [0,0,0] & Initial position of release*  \\
\midrule
movers      & list & []     & List of mover id strings             \\
reactors    & list & []     & List of reactor id strings \\
diffusers   & list & []     & List of diffuser id strings \\
\midrule
stickers    & list & []     & List of sticker id strings \\
unstick     & boolean & 0     & \\
\midrule
tstart      & datetime/int & 0 & Starting time for release \\
tend        & datetime/int & 1.e10 & Ending time for release \\
tstep       & float & 0. & \hl{Timestep of release} \\
\hl{tstep\_release} & float & 0. & \\
nbuff       & int  &        & Total number of particles in buffer  \\
\hl{spawn}       & int & 1 &  Number of spawned particles (per day)  \\
reln        & int & 0 & Number of particles per release  \\
R0          & float & 1. & Total release of material  \\
Q0          & float & 1. & Flux of material (per day)  \\
\midrule
is3d        & boolean & True  & \\
geod        & boolean & False  & \\
\bottomrule
\end{tabular}
\end{table}

\FloatBarrier



\subsection{PassiveTracer}
\label{ssec:passive}

The most simple material is the inert passive tracer (\texttt{class PassiveTracer}), which can only be advected and diffused, without other sinks and sources. \texttt{PassiveTracer} particles enter the computational pool by being released, and can leave by either being transported out of the spatial domain or by interception with shoreline or depth ("stickers", see section \ref{sec:stickers}).


\end{document}